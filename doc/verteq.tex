% -*- mode: latex; mode: visible; TeX-master: t -*-
\documentclass[12pt]{scrartcl}
% --- packages ---
\usepackage{amsmath}
\usepackage{amssymb}
\usepackage{calc}
\usepackage{enumitem}
\usepackage{listings}
\usepackage{lmodern}
\usepackage{mathtools}
\usepackage{refstyle}
\usepackage{varioref}
\usepackage{xspace}

% see http://tex.stackexchange.com/a/23705/12997
\setlist[description,1]{labelindent=\parindent,leftmargin=!}
\lstloadlanguages{Matlab}
\lstset{basicstyle=\ttfamily}

% --- macros ---
% enable use of equations both inline and in equations
\newcommand{\mth}[1]{\ensuremath{#1}\xspace}
\newcommand{\wid}[1]{\widthof{\bfseries {#1} \hspace{\labelsep}}}
\newcommand{\COO}{\mth{\mathrm{CO}_2}}
\newcommand{\MRST}{{\tt MRST}}
\newcommand{\OPM}{{\tt OPM}}

% spacing around function argument parenthesis
% (see http://tex.stackexchange.com/a/2610/12997 and
%      http://tex.stackexchange.com/questions/45218 )
\newcommand{\aleft}{\mathopen{}\mathclose\bgroup\left}  % (
\newcommand{\aright}{\aftergroup\egroup\right}          % )
\newcommand{\binop}[3]{#1\mkern1mu{#2}\mkern1mu #3}     % x - y
\newcommand{\dual}[1]{\binop{1}{-}{#1}}                 % 1 - x
\newcommand{\ddual}[2]{\binop{\binop{1}{-}{#1}}{-}{#2}} % 1 - x - y
\newcommand{\at}[2]{\left[ #1 \right]_{#2}}             % f (x)

% --- notation ---
\newcommand{\der}[2]{\frac{\partial{#1}}{\partial{#2}}}       % derivative
\newcommand{\dvg}[1]{\nabla\cdot{#1}}                         % divergence
\newcommand{\intg}[4]{\int_{#1}^{#2}\!{#3}\,\mathrm{d}{#4}}   % integral
\newcommand{\inv}[1]{{#1}^{-1}}                               % inverse
\newcommand{\latr}[1]{{#1}_{\|}}                              % lateral
\newcommand{\magn}[1]{{#1}_|}                                 % magnitude
\newcommand{\grad}[1]{\nabla #1}                              % gradient
\newcommand{\inner}[2]{\binop{#1}{\cdot}{#2}}                 % inner product
\newcommand{\vect}[1]{\underline{\mathbf{#1}}}                % vector

% --- nomenclature ---
\newcommand{\Hei}{H}        % total height
\newcommand{\hei}{h}        % height of each phase layer
\newcommand{\Por}{\Phi}     % porosity (coarse)
\newcommand{\por}{\phi}     % porosity (fine)
\newcommand{\Sat}{S}        % saturation (coarse)
\newcommand{\sat}{s}        % saturation (fine)
\newcommand{\sr}[1]{\sat_{#1,r}}    % residual saturation
\newcommand{\Prm}{K}        % abs. perm. (coarse)
\newcommand{\prm}{k}        % abs. perm. (fine)
\newcommand{\Rlp}[1]{K_{r,#1}}  % rel. perm. (coarse)
\newcommand{\rlp}[1]{k_{r,#1}}  % rel. perm. (fine)
\newcommand{\tim}{t}        % time
\newcommand{\ttt}{\tau}     % alt. time (var.)
\newcommand{\Vel}{U}        % velocity
\newcommand{\vel}{u}
\newcommand{\phs}{\alpha}   % generic phase
\newcommand{\Lev}{\zeta}    % level
\newcommand{\Top}{T}        % top
\newcommand{\Bot}{B}        % bottom
\newcommand{\Res}{R}        % residual
\newcommand{\Mob}{M}        % mobile
\newcommand{\dph}{z}        % depth
\newcommand{\nap}{c}        % non-aquous phase (co2)
\newcommand{\wet}{b}        % wetting phase (brine)
\newcommand{\res}{r}        % residual
\newcommand{\x}{x}
\newcommand{\y}{y}
\newcommand{\z}{z}
\newcommand{\avg}[2]{\mathcal{A}\aleft[#1, #2\aright]}  % averaging operator
\newcommand{\f}{f}          % generic function
\newcommand{\h}{h}          % height argument (to operator)
\newcommand{\krnwr}{c}      % rel. perm. for co2 when residual brine
\newcommand{\krwnr}{b}      % rel. perm. for brine when residual co2

\newcommand{\pres}{p}       % fine-scale pressure
\newcommand{\Pres}{P}       % coarse-scale pressure
\newcommand{\cpl}{e}        % capillary pressure indicator
\newcommand{\dif}{\mathrm{cap}} % pressure difference indicator
\newcommand{\dens}{\rho}    % density
\newcommand{\grav}{\vect{g}}% gravity

% --- shortcuts ---
\newcommand{\Satp}{\Sat_\phs}
\newcommand{\satp}{\sat_\phs}
\newcommand{\Satn}{\Sat_\nap}
\newcommand{\satn}{\sat_\nap}
\newcommand{\Satnmax}{\Sat_{\nap,\max}} % sat. when plume is largest
\newcommand{\snr}{\sr{\nap}}    % residual gas sat.
\newcommand{\swr}{\sr{\wet}}    % residual brine sat.
\newcommand{\Velp}{\Vel_\phs}
\newcommand{\velp}{\vel_\phs}
\newcommand{\LevT}{\Lev_\Top}
\newcommand{\LevB}{\Lev_\Bot}
\newcommand{\LevM}{\Lev_\Mob}
\newcommand{\LevR}{\Lev_\Res}
\newcommand{\hnap}{\hei_\nap}           % current extent of plume at loc.
\newcommand{\hnmax}{\hei_{\nap,\max}}   % maximum extent of plume at loc.
\newcommand{\Rlpp}{\Rlp{\phs}}  % generic rel. perm.
\newcommand{\rlpp}{\rlp{\phs}}
\newcommand{\Rlpn}{\Rlp{\nap}}  % gas rel. perm.
\newcommand{\rlpn}{\rlp{\nap}}
\newcommand{\Rlpw}{\Rlp{\wet}}  % brine rel. perm.
\newcommand{\rlpw}{\rlp{\wet}}
\newcommand{\Absprm}{\Prm}          % upscaled abs. perm. used in rel. perm. calc.
\newcommand{\absprm}{\latr{\prm}}   % abs. perm. used in rel. perm. calc.

\newcommand{\pe}{\pres_{\cpl}}      % fine-scale entry pressure
\newcommand{\pg}{\pres_{\nap}}      % CO2 fine-scale pressure
\newcommand{\Pg}{\Pres_{\nap}}      % CO2 coarse-scale pressure
\newcommand{\pw}{\pres_{\wet}}      % brine fine-scale pressure
\newcommand{\Pw}{\Pres_{\wet}}      % brine coarse-scale pressure
\newcommand{\Pc}{\Pres_{\dif}}      % upscaled capillary pressure
\newcommand{\densw}{\dens_{\wet}}   % brine density
\newcommand{\densn}{\dens_{\nap}}   % CO2 density

\begin{document}
We assume that we have two phases, \COO and brine:
\begin{equation}
\phs = \left\{ \nap, \wet \right\}
\end{equation}

\begin{description}[labelwidth=\wid{\( \LevM \)}]
\item[ \( \LevT \) ] \(=\) Level of top of the reservoir
\item[ \( \LevB \) ] \(=\) Level of the bottom of the reservoir
\item[ \( \LevM \) ] \(=\) Level of intersection between \COO and residual region \item[ \( \LevR \) ] \(=\) Level of intersection between residual region and brine
\item[ \( \Hei  \) ] \(=\) Total height of reservoir \(= \LevT - \LevB \)
\end{description}

\begin{equation}
\LevB \leq \LevR \leq \LevM \leq \LevT
\end{equation}

Also the following temporal relation:

\begin{equation}
\LevR \aleft( \tim \aright) = \min \LevM \aleft( \ttt \aright), \qquad 0 \leq \ttt < \tim
\end{equation}

For each of these phases, we will evolve a conservation equation for immiscible flow:
\begin{gather}
\eqlabel{verteq:conserv-eq}
\der{ \left( \Hei \Por \Satp \right) }{ \tim } + \dvg{ \Velp } = 0 \\
\intertext{and the algebraic condition for incompressibility:}
\eqlabel{verteq:incompressible}
\dvg{ \sum_{\phs} { \Velp } } = 0
\intertext{where}
\Hei \Por \Satp = \intg{ \LevB }{ \LevT }{ \por \satp }{ \dph }
\intertext{and}
\Velp = \intg{ \LevB }{ \LevT }{ \latr{ \velp } }{ \dph }
\end{gather}

\section{Upscaled quantities}
\subsection{Operators}
We have the ``lateral'' operator~\( \latr{ \hspace{ 0ex } } \) on a vector:

\begin{equation}
\latr{ \vel } = \begin{pmatrix}
\vel_x \\
\vel_y
\end{pmatrix}, \qquad \mathrm{where\ }
\vel = \begin{pmatrix}
\vel_x \\
\vel_y \\
\vel_z
\end{pmatrix}
\end{equation}

and on a matrix:

\begin{equation}
\latr{ \prm } = \begin{pmatrix}
\prm_{\x\x} & \prm_{\x\y} \\
\prm_{\x\y} & \prm_{\y\y}
\end{pmatrix}, \qquad \mathrm{where\ }
\prm = \begin{pmatrix}
\prm_{\x\x} & \prm_{\x\y} & \prm_{\x\z} \\
\prm_{\x\y} & \prm_{\y\y} & \prm_{\y\z} \\
\prm_{\x\z} & \prm_{\y\z} & \prm_{\z\z}
\end{pmatrix}
\end{equation}

and to get the magnitude of such a lateral tensor, we select the Frobenius norm as suffix operator~\( \magn{ \hspace{ 0ex } } \):

\begin{equation}
\magn{ \prm } = \| \latr{ \prm } \|_F = \sqrt{ \prm_{\x\x}^2 + 2\prm_{\x\y}^2 + \prm_{\y\y}^2 }
\end{equation}

We have the depth-averaging operator:

\begin{equation}
\avg{ \f }{ \h } = \frac{ 1 }{ \Hei } \intg{ \h }{ \LevT }{ \f \aleft( \dph \aright) }{ \dph }
\end{equation}

and its derivative:

\begin{equation}
\der{ \avg{ \f }{ \h } }{ \h } = - \frac{ 1 }{ \Hei } \f \aleft( \h \aright)
\end{equation}

We also use the notation:

\begin{equation}
\at{ \f }{ \h } \equiv \f \aleft( \h \aright)
\end{equation}

when \( \f \) is a composite expression.

\subsection{Porosity}
Porosity is depth-averaged over the column:

\begin{align}
\Hei \Por &= \intg{ \LevB }{ \LevT }{ \por }{ \dph } \\
\Por &= \avg{ \por }{ \LevB }
\end{align}

\subsection{Absolute Permeability}
\begin{align}
\Hei \Prm &= \intg{ \LevB }{ \LevT }{ \latr{ \prm } }{ \dph } \\
\Prm &= \avg{ \latr{ \prm } }{ \LevB }
\end{align}

\subsection{Saturation}
We'll be making the following assumption:

\begin{equation}
\satn =
\begin{cases}
0        & \Leftarrow \LevB \leq \dph < \LevR \\
\snr     & \Leftarrow \LevR \leq \dph < \LevM \\
\dual{ \swr } & \Leftarrow \LevM \leq \dph < \LevT
\end{cases}
\end{equation}

and make use of the depth-averaging operator on these quantities:

\begin{description}[labelwidth=\wid{\( \por \left( \ddual{ \swr }{ \snr } \right) \)}]
\item[ \( \por \left( \snr \right) \) ] Fraction of volume in each block filled with residual gas
\item[ \( \por \left( \ddual{ \swr }{ \snr } \right) \) ] Fraction of volume in each block filled with mobile fluid
\item[ \( \por \left( \dual{ \swr } \right) \) ] Fraction of volume in each block filled with residual brine
\end{description}

leading to the interpretations:

\begin{description}[labelwidth=\wid{\( \avg{ \por \left( \ddual{ \swr }{ \snr } \right) }{ \h } \)}]
\item[ \( \avg{ \por \left( \snr \right) }{ \h } \) ] Fraction of the column, up to \( \h \), filled with residual gas
\item[ \( \avg{ \por \left( \ddual{ \swr }{ \snr } \right) }{ \h } \) ] Fraction of the column, up to \( \h \), filled with mobile fluid
\item[ \( \avg{ \por \left( \dual{ \swr } \right) }{ \h } \) ] Fraction of the column, up to \( \h \), filled with residual brine
\end{description}

When the plume is at its largest \( \LevM = \LevR \), we have:

\begin{align}
\Hei \Por \Satnmax &= \intg{ \LevR }{ \LevT }{ \por \left( \dual{ \swr } \right) }{ \dph } \\
\eqlabel{ve:res-eq}
\Por \Satnmax &= \avg{ \por \left( \dual{ \swr } \right) }{ \LevR }
\end{align}

\begin{align}
\Hei \Por \Satn &= \intg{ \LevB }{ \LevT }{ \por \satn }{ \dph } \\
&= \intg{ \LevB }{ \LevR }{ \por \cdot 0 }{ \dph } + \intg{ \LevR }{ \LevM }{ \por \snr }{ \dph } + \intg{ \LevM }{ \LevT }{ \por \left( \dual{ \swr } \right) }{ \dph } \\
&= \intg{ \LevR }{ \LevT }{ \por \snr }{ \dph } - \intg{ \LevM }{ \LevT }{ \por \snr }{ \dph } + \intg{ \LevM }{ \LevT }{ \por \left( \dual{ \swr } \right) }{ \dph } \\
&= \intg{ \LevR }{ \LevT }{ \por \snr }{ \dph } + \intg{ \LevM }{ \LevT }{ \por \left( \ddual{ \swr }{ \snr } \right) }{ \dph }
\end{align}

\begin{align}
\Por \Satn - \avg{ \por \snr }{ \LevR } &= \avg{ \por \left( \ddual{ \swr }{ \snr } \right) }{ \LevM }
\end{align}

Given that \( \LevR \) is known, we can solve the above equation to find \( \LevM \). (What if \( \LevM \aleft( \tim \aright) < \LevR \aleft( \tim - 1 \aright) \)?)

We know that:

\begin{equation}
\Por \der{ \Satn }{ \LevM } = \der{ \avg{ \por \left( \ddual{ \swr }{ \snr } \right) }{ \LevM } }{ \LevM } = - \frac{ \at{ \por \left( \ddual{ \swr }{ \snr } \right) }{ \LevM } }{ \Hei }
\end{equation}

\subsection{Allowable Range}

Since the upscaled saturation is now an expression for height, we have:

\begin{equation}
0 \leq \Satn \leq 1
\end{equation}

\subsection{Relative Permeability}
\begin{equation}
\Hei \Prm \Rlpp = \intg{ \LevB }{ \LevT }{ \latr{ \prm } \rlpp }{ \dph }
\end{equation}

Notice that the argument of the rel.perm. function on the right-hand side is not the same as the argument of the operator of the function on the left-hand side.

Instead of evaluating the fine-scale rel.perm. function, we'll make the following assumption:

\begin{align}
\rlpn \aleft( \satn \aright) &=
\begin{cases}
\rlpn \aleft( 0 \aright) = 0 & \Leftarrow \LevB \leq \dph < \LevR \\
\rlpn \aleft( \snr \aright) = 0 & \Leftarrow \LevR \leq \dph < \LevM \\
\rlpn \aleft( \dual{\swr} \aright) = \krnwr & \Leftarrow \LevM \leq \dph < \LevT
\end{cases}
\end{align}

yielding:

\begin{align}
\Hei \Absprm \Rlpn &= \intg{ \LevB }{ \LevR }{ \absprm \cdot 0 }{ \dph } + \intg{ \LevR }{ \LevM }{ \absprm \cdot 0 }{ \dph } + \intg{ \LevM }{ \LevT }{ \absprm \cdot \krnwr }{ \dph } \\
\Rlpn \aleft( \Satn \aright) &= \inv{ \Absprm } \avg{ \absprm \cdot \krnwr }{ \LevM }
\end{align}

with the derivative:

\begin{align}
\der{ \Rlpn \aleft( \Satn \aright) }{ \Satn } &= \inv{ \Absprm } \der{ \avg{ \absprm \cdot \krnwr }{ \LevM } }{ \LevM } \der{ \LevM }{ \Satn } \\
&= \inv{ \Absprm } \frac{ \at{ \absprm \cdot \krnwr }{ \LevM }}{ \Hei } \frac{ \Hei \Por }{ \at{ \por \left( \ddual{ \swr }{ \snr } \right) }{ \LevM }} \\
&= \Por \inv{ \Absprm } \at{ \frac{ \absprm \cdot \krnwr }{ \por \left( \ddual{ \swr }{ \snr } \right) } }{ \LevM }
\end{align}

And then for brine:

\begin{align}
\rlpw \aleft( \satn \aright) &=
\begin{cases}
\rlpn \aleft( 0 \aright) = 1 & \Leftarrow \LevB \leq \dph < \LevR \\
\rlpw \aleft( \snr \aright) = \krwnr & \Leftarrow \LevR \leq \dph < \LevM \\
\rlpw \aleft( \dual{ \swr } \aright) = 0 & \Leftarrow \LevM \leq \dph < \LevT
\end{cases}
\end{align}

subsequently:

\begin{align}
\Hei \Absprm \Rlpw &= \intg{ \LevB }{ \LevR }{ \absprm \cdot 1 }{ \dph } + \intg{ \LevR }{ \LevM }{ \absprm \cdot \krwnr }{ \dph } + \intg{ \LevM }{ \LevT }{ \absprm \cdot 0 }{ \dph } \\
&= \intg{ \LevB }{ \LevT }{ \absprm }{ \dph } - \intg{ \LevR }{ \LevT }{ \absprm }{ \dph } + \intg{ \LevR }{ \LevT }{ \absprm \cdot \krwnr }{ \dph } - \intg{ \LevM }{ \LevT }{ \absprm \cdot \krwnr }{ \dph } \\
\Rlpw \aleft( \Satn \aright) &= \inv{ \Absprm } \left( \avg{ \absprm }{ \LevB } - \avg{ \absprm \cdot \left( \dual{ \krwnr } \right) }{ \LevR } - \avg{ \absprm \cdot \krwnr }{ \LevM } \right) \\
&= I - \inv{ \Absprm } \left( \avg{ \absprm \cdot \left( \dual{ \krwnr } \right) }{ \LevR } + \avg{ \absprm \cdot \krwnr }{ \LevM } \right)
\end{align}

with

\begin{align}
\der{ \Rlpw \aleft( \Satn \aright) }{ \Satn } &= - \inv{ \Absprm } \der{ \avg{ \absprm \cdot \krwnr }{ \LevM } }{ \LevM } \der{ \LevM }{ \Satn } \\
&= - \Por \inv{ \Absprm } \frac{ \absprm \cdot \krwnr }{ \por \left( \ddual{ \swr }{ \snr } \right) }
\end{align}

Thus, we have the interpretation of the values

\begin{description}[labelwidth=\wid{\( \krnwr \)}]
\item[ \( \krwnr \) ] Maximum mobility of brine when \COO is present
\item[ \( \krnwr \) ] Maximum mobility of \COO when brine is present (always)
\end{description}

Notice that the relative permeabilities are now tensors, since they are weighted sums of other tensors. However, simulation code usually expects a scalar for this. We work around this by creating a scalar that tells us something about the magnitude of permeability We therefore substitute \( \latr{ \prm } \) with \( \magn{ \prm } \).

\subsection{Implementation}

We use the following variable names for expressions, in \OPM:

\begin{description}[labelwidth=\wid{\( \intg{ \h }{ \LevM }{ \por \left( \ddual{ \swr }{ \snr } \right) }{ \dph } \)}]
\item[ \( \rlpn \aleft( \dual{\swr} \aright) = \krnwr \) ] \lstinline[language=C++]!gas_mob[GAS]!
\item[ \( \rlpw \aleft( \snr \aright) = \krwnr \) ] \lstinline[language=C++]!wat_mob[WAT]!
\item[ \( \por \snr \) ] \lstinline[language=C++]!res_gas_vol!
\item[ \( \por \left( \ddual{ \swr }{ \snr } \right) \) ] \lstinline[language=C++]!mob_mix_vol!
\item[ \( \por \left( \dual{ \swr } \right) \) ] \lstinline[language=C++]!res_wat_vol!
\item[ \( \intg{ \h }{ \LevM }{ \por \snr }{ \dph } \) ] \lstinline[language=C++]!res_gas_dpt!
\item[ \( \intg{ \h }{ \LevM }{ \por \left( \ddual{ \swr }{ \snr } \right) }{ \dph } \) ] \lstinline[language=C++]!mob_mix_dpt!
\item[ \( \intg{ \h }{ \LevM }{ \por \left( \dual{ \swr } \right) }{ \dph } \) ] \lstinline[language=C++]!res_wat_dpt!
\item[ \( \magn{ \prm } \) ] \lstinline[language=C++]!lkl!
\item[ \( \inv{ \Prm } \) ] \lstinline[language=C++]!1/tot_lkl!
\item[ \( \inv{ \Prm } \latr{ \prm } \cdot \left( \dual{ \krwnr } \right) \) ] \lstinline[language=C++]!prm_wat!
\item[ \( \inv{ \Prm } \intg{ \h }{ \LevM }{ \latr{ \prm } \cdot \left( \dual{ \krwnr } \right) }{ \dph } \) ] \lstinline[language=C++]!prm_wat_int!
\item[ \( \inv{ \Prm } \latr{ \prm } \cdot \krwnr \) ] \lstinline[language=C++]!prm_res!
\item[ \( \inv{ \Prm } \intg{ \h }{ \LevM }{ \latr{ \prm } \cdot \krwnr }{ \dph } \) ] \lstinline[language=C++]!prm_res_int!
\item[ \( \inv{ \Prm } \latr{ \prm } \cdot \krnwr \) ] \lstinline[language=C++]!prm_gas!
\item[ \( \inv{ \Prm } \intg{ \h }{ \LevM }{ \latr{ \prm } \cdot \krnwr }{ \dph } \) ] \lstinline[language=C++]!prm_gas_int!
\end{description}

\subsection{Simplification}

Notice the special case:

\begin{align}
\avg{ 1 }{ \h } = \frac{ \LevT - \h }{ \Hei } & & \mathrm{ and } & & \der{ \avg{ 1 }{ \h } }{ \h } = - \frac{ 1 }{ \Hei }
\end{align}

Instead of levels, we operate directly on heights:

\begin{description}[labelwidth=\wid{\( \hnmax \)}]
\item[ \( \hnap \) ] = Height of \COO plume = \( \LevT - \LevM \)
\item[ \( \hnmax \) ] = Height of plume \emph{and} residual region = \( \LevT - \LevR \)
\end{description}

If we assume that we have a homogeneous layer where:

\begin{align}
\der{ \por }{ \dph } &= 0 \\
\der{ \prm }{ \dph } &= 0 \\
\der{ \rlpp }{ \dph } &= 0
\end{align}

we get

\begin{align}
\Por \prime &= \avg{ 1 }{ \LevB } = 1 \\
\Absprm \prime &= \avg{ 1 }{ \LevB } = 1
\end{align}

The relation~\eqref[vref]{ve:res-eq} between the maximum upscaled gas saturation and the maximum plume height becomes:

\begin{align}
\Satnmax \prime &= \frac{ \left( \dual{ \swr } \right) \cdot \left( \LevT - \LevR \right) }{ \Hei } \\
&= \frac{ \left( \dual{ \swr } \right) \hnmax }{ \Hei } \\
\hnmax &= \frac{ \Hei \Satnmax \prime }{ \dual{ \swr } }
\end{align}

This is ``\lstinline[language=Matlab]!s_max*H=h_max*(1-sr(2))!'' in \MRST.

The equation to be solved to find the interface now becomes:

\begin{align}
\Satn \prime - \avg{ \snr }{ \LevR } &= \avg{ \ddual{ \swr }{ \snr } }{ \LevM } \\
\Satn \prime - \frac{ \snr \hnmax }{ \Hei } &= \frac{ \left( \ddual{ \swr }{ \snr } \right) \hnap }{ \Hei }
\end{align}

with

\begin{align}
\der{ \Satn \prime }{ \hnap } &= \frac{ \ddual{ \swr }{ \snr } }{ \Hei }
\end{align}

In \MRST we have \lstinline[language=Matlab]!dh=bsxfun(@rdivide,opt.H,1-opt.sr(2)-opt.sr(1))!, so this is \lstinline[language=Matlab]!1/dh!.

We solve the equation to find \( \hnap \):
\begin{align}
\hnap &= \frac{ \Hei \Satn \prime - \snr \hnmax }{ \ddual{ \swr }{ \snr } } \\
&= \frac{ \Hei \Satn \prime - \frac{ \snr \Hei \Satnmax \prime }{ \dual{ \snr } } }{ \ddual{ \swr }{ \snr } } \\
&= \frac{ \Hei }{ \ddual{ \swr }{ \snr } } \left( \Satn \prime - \frac{ \snr }{ \dual{ \snr } } \Satnmax \prime \right)
\end{align}

This is ``\lstinline[language=Matlab]!s*H=h*(1-sr(2))+(h_max-h)*sr(1)!'' in \MRST.

Thus

\begin{align}
\Rlpn \prime &= \avg{ \krnwr }{ \LevM } \\
\der{ \Rlpn \prime }{ \Satn } &= \at{ \frac{ \krnwr }{ \ddual{ \swr }{ \snr } } }{ \LevM }
\end{align}

which is represented by \lstinline[language=Matlab]!bsxfun(@rdivide,dh*opt.kwm(1),g_top.H)! (there is implicitly a \lstinline[language=Matlab]!H! in \lstinline[language=Matlab]!dh!) in \MRST.

And

\begin{align}
\Rlpw \prime &= 1 - \avg{ \dual{ \krwnr } }{ \LevR } - \avg{ \krwnr }{ \LevM } \\
\der{ \Rlpw \prime }{ \Satn } &= - \frac{ \krwnr }{ \ddual{ \swr }{ \snr } }
\end{align}

\subsection{Pressure}
We use the pressure on the \emph{top surface} of the column as the upscaled pressure for \COO as this is the point where there supposedly is a largest chance of encountering a mobile phase. If we have two pressure unknowns, we could consequently also select the \emph{bottom} as the reference point for brine, but since we have only one unknown for the pressure in the underlaying simulator, it should be picked from the same point for brine as for \COO. We thus have the reference point:
\begin{align}
\eqlabel{ve:pc:ref-pnt}
\Pres_\phs = \pres_\phs \aleft( \LevT \aright)
\end{align}

Pressure downward in the column is assumed to be in hydrostatic equilibrium for both phases:
\begin{align}
\eqlabel{ve:pc:hydro-static}
\pres_\phs \aleft( \h \aright) &= \Pres_\phs + \inner{ \grad{ \dph } }{ \grav } \intg{ \h }{ \LevT }{ \dens_{\phs} }{ \dph }
\end{align}

\subsection{Capillary Pressure}
In \emph{equilibrium} the difference between the pressures should be equal to the entry pressure

\begin{equation}
\eqlabel{ve:pc:entry-press}
\pe \aleft( \satn \aleft( \h \aright) \aright) = \pg \aleft( \h \aright) - \pw \aleft( \h \aright)
\end{equation}

If we now insert \eqref[vref]{ve:pc:hydro-static} into each of the terms  on the right-hand side of \eqref[vref]{ve:pc:entry-press}, we get

\begin{equation}
\eqlabel{ve:pc:hydrostatic-entry}
\pe \aleft( \satn \aleft( \h \aright) \aright) = \Pg - \Pw + \inner{ \grad{ \dph } }{ \grav } \intg{ \h }{ \LevT }{ \dens_{\nap} - \dens_{\wet} }{ \dph }
\end{equation}

The level where there is equilibrium between the phases is in the interface between them, i.e.~where \( \h = \LevM \). If we insert this into \eqref[vref]{ve:pc:hydrostatic-entry} we get

\begin{equation}
\eqlabel{ve:pc:upscaled-press-diff}
\Pc \aleft( \LevM \aright) = \pe \aleft( \at{ \satn }{ \LevM } \aright) - \inner{ \grad{ \dph } }{ \grav } \intg{ \LevM }{ \LevT }{ \dens_{\nap} - \dens_{\wet} }{ \dph }
\end{equation}

Since we assume incompressible fluids and thus constant densities, we can use the write out the integrals on the right-hand side using the levels themselves:

\begin{equation}
\eqlabel{ve:pc:incomp-cap}
\Pc \aleft( \LevM \aright) = \pe \aleft( \at{ \satn }{ \LevM } \aright) - \inner{ \grad{ \dph } }{ \grav } \left( \dens_{\nap} - \dens_{\wet} \right) \left( \LevT - \LevM \right)
\end{equation}

We can take the derivative of \eqref[vref]{ve:pc:upscaled-press-diff}:
\begin{equation}
\eqlabel{ve:pc:cap-deriv}
\der{ \Pc }{ \LevM } = \der{ \pe }{ \satn } \der{ \satn }{ \LevM } + \inner{ \grad{ \dph } }{ \grav } \left( \dens_{\nap} - \dens_{\wet} \right)
\end{equation}

(The sign change is because we take the derivative with respect to the lower bound).

Note that in the discretized version we have
\begin{equation}
\eqlabel{ve:pc:sat-deriv}
\der{ \satn }{ \LevM } = \frac{ 1 }{ \at{ \mathrm{d}\dph }{ \LevM } }
\end{equation}
\end{document}
